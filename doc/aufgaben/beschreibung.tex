\documentclass[oneside,a4paper]{scrartcl}

\input{/home/johannes/documents/cedarsoft/korrespondenz/vorlagen/tex/common.tex}


\author{Johannes Schneider}

\begin{document}


\centerline{\sc \large Aufgabenbeschreibung Live-Aufgabe}
\vspace{.5pc}
\centerline{\sc Objekttechnologien - Sommersemester 2016}
\vspace{2pc}


\section{Grundlegendes}

Im Rahmen dieser Live-Aufgabe soll mit Hilfe des Abstract-Factory-Patterns 
die \enquote{Erzeugung} von Fußball-Spielern abgebildet werden.


\section{Erstellen Sie folgende Klassen/Interfaces}
\subsection{Jersey}

Eine Trikot enthält folgende Eigenschaften

\begin{description}
  \item[color] Die Farbe des Trikots (als java.awt.Color oder notfalls als String)
  \item[number] Die Trikot-Nummer (als int)
\end{description}

\subsection{JerseyWithPadding}
Erweitert die Klasse Jersey.

Entspricht einem Trikot mit Polsterung (für einen Torwart).
Hat im Vergleich zu dem normalen Jersey der Einfachheit halber keine weiteren Felder.

\subsection{Trunks}
Eine Sporthose enthält folgende Eigenschaft:

\begin{description}
  \item[color] Die Farbe des Trikots (als java.awt.Color oder notfalls als String)
\end{description}


\section{Abstract-Factory-Pattern}

\subsection{AbstractPlayerFactory}

Mit Hilfe zweier Implementierungen von AbstractPlayerFactory soll es möglich sein, einen Feldspieler (mit der FielderFactory) bzw.
einen Torwart (mit der GoalieFactory) zu erzeugen.

Für jeden Spieler sollen dabei über die jeweilige Factory folgende Dinge erzeugt werden können:

\begin{description}
  \item[jersey] Ein Trikot: Für Feldspieler in der Farbe weiß, für den Torwart in grün.
  \item[trunks] Eine Hose: Für Feldspieler in der Farbe schwarz, für den Torwart in grün.
\end{description}



\end{document}

