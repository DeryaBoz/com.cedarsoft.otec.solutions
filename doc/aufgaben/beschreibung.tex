\documentclass[oneside,a4paper]{scrartcl}

\input{/home/johannes/documents/cedarsoft/korrespondenz/vorlagen/tex/common.tex}


\author{Johannes Schneider}

\begin{document}


\centerline{\sc \large Aufgabenbeschreibung Live-Aufgabe}
\vspace{.5pc}
\centerline{\sc Objekttechnologien - Wintersemester 2015/2016}
\vspace{2pc}



\section{Grundlegendes}

Im Rahmen dieser Live-Aufgabe soll mit Hilfe des Builder-Patterns eine (einfache) Bestellung in einem Fast-Food-Restaurant abgebildet werden.


\section{Erstellen Sie folgende Klassen/Interfaces}
\subsection{Order}

Eine Bestellung enthält folgenden Felder mit zugehörigen Getter-Methoden:

\begin{description}
  \item[price] Der Gesamt-Preis der Bestellung (ausnahmsweise als einfacher Zahlenwert)
  \item[burgers] Eine Liste von Strings, welche die bestellten Burger repräsentieren
  \item[drinks] Eine List von Strings, welche die bestellten Getränke repräsentieren
\end{description}


\section{Builder-Pattern}

\subsection{OrderBuilder}

Mit Hilfe des OrderBuilders soll es möglich sein, bequem eine Bestellung zu erstellen.

Der Builder soll dabei folgende Funktionalitäten abbilden:

\subsubsection{Hinzufügen einzelner Burger}
Einzelne Burger sollen (einzeln) zum OrderBuilder hinzugefügt werden können.

\subsubsection{Hinzufügen einzelner Drinks}
Einzelne Drings sollen (einzeln) zum OrderBuilder hinzugefügt werden können.

\subsubsection{Setzen eines Preises}
Der Gesamt-Preis der Bestellung soll gesetzt werden können.

\subsubsection{Erstellen einer Order}
Eine Bestellung soll aus den im Builder gespeicherten Werten erzeugt werden. 

Bei der Erzeugung der Order soll automatisch eine Plausibilitäts-Prüfung durchgeführt werden.

\subsubsection{Plausibilitäts-Prüfung}
Überprüft werden soll dabei folgendes:

\begin{itemize}
  \item Der Preis muss positiv sein
  \item Es muss mindestens ein Burger oder ein Drink bestellt werden.
\end{itemize}



\subsection{Unit-Test / Main-Methode}
Erstellen Sie (mindestens) einen Unit-Test oder (mindestens) eine Main-Klasse welches das Verhalten des Builders verdeutlicht.

Dabei soll auch die Funktionsfähigkeit der Validierung gezeigt werden.


\end{document}

