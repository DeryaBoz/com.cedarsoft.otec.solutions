\documentclass[oneside,a4paper]{scrartcl}

\input{/home/johannes/documents/cedarsoft/korrespondenz/vorlagen/tex/common.tex}


\author{Johannes Schneider}

\begin{document}


\centerline{\sc \large Aufgabenbeschreibung Live-Aufgabe}
\vspace{.5pc}
\centerline{\sc Objekttechnologien - Wintersemester 2014/2015}
\vspace{2pc}



\section{Grundlegendes}

Im Rahmen dieser Live-Aufgabe soll mit Hilfe des Visitor-Patterns ein \enquote{Hacking-Tool} für
unterschiedliche Router-Typen erstellt werden:


\section{Erstellen Sie folgende Klassen/Interfaces}
\subsection{Interface Router}

Ein Router enthält die Methode \enquote{void sendData(String data);}.

\subsection{Drei Router-Implementierungen}
Erstellen Sie drei Router-Implementierungen (z.B. benannt nach unterschiedlichen Herstellern). Diese
implementieren die \#sendData(String)-Methode zunächst identisch:

Die übergebenen Daten werden auf der Konsole ausgegeben.

\subsubsection{Sicherheitslücken}
Jeder der drei Router-Implementierungen erhält nun eine Sicherheitslücke:
Wenn ein bestimmer (von Ihnen frei gewählter) String an den Router geschickt wird, 
gilt dieser als gehackt. Dies wird durch eine entsprechende Ausgabe auf der Konsole verdeutlicht.

Erweitern Sie also die Methoden-Implementierungen der drei Router-Klassen und geben Sie jeweils eine entsprechend Ausgabe auf der Konsole
aus, sofern der herstellerspezifischer String übergeben wird.


\section{Visitor-Pattern}

\subsection{HackingVisitor}

Mit Hilfe des Visitor-Patterns soll nun \enquote{HackingVisitor} erstellt werden.
Dieser Visitor übergibt je nach Hersteller den bestimmten String an die sendData(String)-Methode und
hackt somit diesen Router.


\subsection{Unit-Test / Main-Methode}
Erstellen Sie einen Unit-Test oder eine Main-Klasse welches das Verhalten des Visitors verdeutlicht.



\end{document}

